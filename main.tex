\documentclass[12pt]{amsart}

%%%%%%%%%%%%%%%%%%%%%%%%
%% project modularity %%
%%
\usepackage[subpreambles=true]{standalone}
\usepackage{import} % more elegant than \input for standalone documents

%%%%%%%%%%%%%%%%%%%%%%%%%%%%%%%%%%
%% design, formatting and fonts %%
%%

%\usepackage[titletoc]{appendix}
%\usepackage[toc,page]{appendix}
\usepackage{amsmath,amssymb,amsthm,amsrefs}
\usepackage{mathtools} % TODO find better for \coloneqq (':=')
\usepackage{mathpazo}
\usepackage{inconsolata}
%\usepackage{euler}
\usepackage{epigraph} % TODO how does this work
\usepackage{showkeys} % TODO http://texdoc.net/texmf-dist/doc/latex/tools/showkeys.pdf
\usepackage{etoolbox}
\usepackage{ifthen} % one route to argument overloading

%%%%%%%%%%%%%%%%%%
%% Nomenclature %%
%%
%\usepackage[intoc]{nomencl}
\usepackage{nomencl}
\makenomenclature

%%%%%%%%%%%%%%
%% graphics %%
%%
%% TODO https://www.sharelatex.com/learn/Inserting_Images
\usepackage{graphicx}
\usepackage{float}
\graphicspath{{graphics/}{../graphics/}} % relative to both / and sections/

%%%%%%%%%%%%%%%%%%%%%%%%%%%%%%%%%%%
%% TODO learn correct usage of tikz
%% tikz
\usepackage{tikz-cd}
%% Functions
%% \begin{tikzcd}[column sep= small,row sep=0ex]
%%     M_f \colon \pi_1(S \smallsetminus \{y_1, \dots, y_n\}, y) \arrow[r]& \Bij(f^{-1}(y)) \\
%%    \gamma \arrow[r, mapsto]                                   & M_f(\gamma) = \sigma_{\gamma}^{-1}
%% \end{tikzcd}
%%
%% \begin{align*}
%%   M_f \colon \pi_1(S \smallsetminus \{y_1, \dots, y_n\}, y) & \longrightarrow \Bij(f^{-1}(y)) \\
%%   \gamma & \longmapsto M_f(\gamma) = \sigma_\gamma^{-1}
%% \end{align*}
%%
%% Or this:\medskip
%%
%% Let $ S' =S \smallsetminus \{y_1, \dots, y_n\} $. Define
%% $ \begin{aligned}[t]
%% M_f \colon \pi_1(S', y) &\longrightarrow \Bij(f^{-1}(y)) \\
%% \gamma &\longmapsto M_f(\gamma) = \sigma_\gamma^{-1}
%% \end{aligned} $


%%%%%%%%%%%%%%%%%%%%%%%%%%%%%%%%%%%%%%%%%%%%%%%%%%
%% TODO find a better way to handle subsections %%
%% - glg.tex file inclusions
%% - introduce nomenclature db using section tags + groups
%% - standardize labels throughout project
%% - place the above into an array and autogen glg.tex
%% - glg.tex file inclusions
%% - introduce nomenclature db using section tags + groups
%% - standardize labels throughout project

% \usepackage{etoolbox}
%  \renewcommand\nomgroup[1]{%
%    \item[\bfseries
%      \ifstrequal{#1}{\leca}{\lecatitle}{%
%      \ifstrequal{#1}{\lecb}{\lecbtitle}{%
%      \ifstrequal{#1}{\lecc}{\lecctitle}{%
%      }}}%
%   ]}
%%%%%%%%%%%%%%%%%%%%%%%
%% Table of contents %%
%%
%% TODO
%% - fix nomenclature
%%
%% \makeatletter
%% \def\thenomenclature{%
%%   \section*{\nomname}
%%   \if@intoc\addcontentsline{toc}{section}{\nomname}\fi%
%% \nompreamble
%% \list{}{%
%% \labelwidth\nom@tempdim
%% \leftmargin\labelwidth
%% \advance\leftmargin\labelsep
%% \itemsep\nomitemsep
%% \let\makelabel\nomlabel}}
%% \makeatother


%%%%%%%%%%%%%%%%%%%%%%%%%%
%% Formatting / Display %%
%%
% \newcommand{\HRule}{\rule{\linewidth}{0.5mm}}
\numberwithin{equation}{section}
% TODO understand what this does
% (something like number equations within sections)

%%%%%%%%%%%%%
%% urls/links
% \usepackage{hyperref}

% examples
% c.f. \hyperref[mainlemma1]{lemma \ref*{mainlemma} }.
% take a look at my website \url{http://raeez.com}
% it never hurts to \href{http://wiki.org/RTFM}{RTFM}
% I can be reached at
% \href{mailto:this_is_a_false_addr@raeez.com}{this\_is\_a\_false\_addr@raeez.com}

%%%%%%%%%%%%%%%%%%%%%%%%%%
%% Theorem Environments %%
%%
\newtheorem{thm}{Theorem}[section]
\newtheorem{prop}[thm]{Proposition}
\newtheorem{lem}[thm]{Lemma}
\newtheorem{cor}[thm]{Corollary}
\theoremstyle{remark}
\newtheorem{rmk}[thm]{Remark}
\theoremstyle{definition}
\newtheorem{defn}[thm]{Definition}
\newtheorem{ex}[thm]{Example}
\newtheorem{exc}[thm]{Exercise}
\newtheorem{conj}[thm]{Conjecture}
\newtheorem{prob}[thm]{Problem}
\newtheorem{oprob}[thm]{Open Problem}
\newtheorem{stmt}[thm]{Statement}
\newtheorem{qn}[thm]{Question}
\newtheorem{ans}[thm]{Answer}

%%%%%%%%%%%%%%%%%%
%% Nomenclature %%
%%
%% TODO figure out better solution
%% https://tex.stackexchange.com/questions/361373/nomenclature-entry-in-toc-not-indented-like-a-chapter/361376

%%%%%%%%%%%%%%%%%%%%%%%%%%%%%%%%%%%%
%% Modify nomenclature generation %%
%% 1. SI Units
%% 2. titled groups
%% 3. enforce manual order

%% 1. Enable SI units
%% \usepackage{siunitx}
%% \sisetup{
%% inter-unit-product=\ensuremath{{}\cdot{}},
%% per-mode=symbol
%% }
%% \nc{\nomunit}[1]{\renewcommand{\nomentryend}{\hspace*{\fill}#1}}

%% c.f. https://tex.stackexchange.com/questions/118114/commands-that-may-take-a-variable-number-of-arguments


%% 2. titled groups

%% TODO 3. manual ordering

%% TODO why is % often used before a newline?

%% TIP wrap long descriptions in a \parbox e.g.
%% \nm[x]{$x$}{\parbox[t]{.75\textwidth}{Unknown variable with a very very
%% very very very very very very long description}\nomunit{\si{\second}}}

%% The following implements grouping in the nomenclature preamble
%% c.f. 1. https://tex.stackexchange.com/questions/166556/how-to-make-a-clean-and-grouped-nomenclature-list
%%      2. https://tex.stackexchange.com/questions/310128/grouped-nomenclature
%%      3. https://tex.stackexchange.com/questions/318850/grouping-nomenclature-elements
%%      4. https://www.sharelatex.com/learn/Nomenclatures

%% \begin{document}
%% \mbox{}
%%
%% \nm[A, 02]{$c$}{Speed of light in a vacuum inertial system}
%% \nm[A, 03]{$h$}{Plank Constant}
%% \nm[A, 01]{$g$}{Gravitational Constant}
%% \nm[B, 03]{$\mathbb{R}$}{Real Numbers}
%% \nm[B, 02]{$\mathbb{C}$}{Complex Numbers}
%% \nm[B, 01]{$\mathbb{H}$}{Octonions}
%% \nm[C]{$V$}{Constant Volume}
%% \nm[C]{$\rho$}{Friction Index}

%%%%%%%%%%%%%%%%%%%%%%%%
%% Markup Readability %%
%%
%% macros
\newcommand\nc{\newcommand}
\newcommand\rc{\renewcommand}

%% nomenclature
%\newcommand\nm{\nomenclature}
\newcommand\nm{} % temporarily disable standard \nomenclature
%\newcommand\mathnm[2]{\nm[#1]{}}

%% shorten the macro syntax
\newcommand\mc[1]{\mathcal{#1}}
\newcommand\mbb[1]{\mathbb{#1}}
\newcommand\mbf[1]{\mathbb{#1}}
\newcommand\mrm[1]{\mathrm{#1}}

\newcommand\image[1]{
  \begin{figure}[H]\label{fig-#1}
    \centering
    \includegraphics[scale=.5]{#1}
  \end{figure}
}
\newcommand\imageopt[3]{ % \imageopt #1-filename #2-scale #3-caption
  \begin{figure}[H]
    \centering
    \ifstrequal{#2}{}{%if no scale provided
      \includegraphics{#1}}{ % don't set the scale
      }{\includegraphics[scale=#2]{#1}}\label{fig:#1}
    \ifstrequal{#3}{}{%if no caption provided
      }{ % don't try to set one
      }{\caption{#3}} % otherwise do
  \end{figure}}

\newcommand\mathsc[1]{\text{\normalfont\scshape#1}}

%% common symbols
\newcommand\lb{\left[}  \newcommand\rb{\right]}  % [ ]
\newcommand\llb{\lb\lb}  \newcommand\rrb{\rb\rb} % [[ ]]

\newcommand\lp{\left(}  \newcommand\rp{\right)} % ( )
\newcommand\llp{\lp\lp} \newcommand\rrp{\rp\rp} % (( ))

\newcommand\twotuple[2]{\lp #1,#2\rp}   % TODO implement variable number of args
\newcommand\setpresent[2]{\{ #1 | #2\}} % TODO typeset correctly

%% Index into Symbols and Notation
%% We collect all notation utilized throughouth this lecture series
%% we the hope their collation facillitates future functional/aesthetic changes

%%%%%%%%%%%%%%%%%%%%%%%%%%%%%%%%%%%%
%% TODO
%% - clarify class formation group k
%%    * little/big, more readable
%%    * visually distinguish notation for C_L / element
%% - Generate a glossary
%% - mark up structure for nomenclature / glossary inline
%%

\nc\Cat[1]{\mathbf{#1}} % typeset name of category
% TODO fit these in
\nc\G{G}           % reductive group
% algebraic fields, formal power series and their fraction fields
\nc\ringint{\mathcal{O}}           % abstract ring of integers
\nc\grk{k}                         %abstract ground field
\nc\N{\mathbb{N}}
\nc\Z{\mathbb{Z}}
\nc\Q{\mathbb{Q}}
\nc\R{\mathbb{R}}
\nc\C{\mathbb{C}}
\nc{\FnS}{\grk\lp\Sigma\rp}
\nc\laurentpoly{\lb x,x^{-1}\rb}
\nc\laurentser{\llp~z\rrp}
\nc\fpowerser{\llb~z\rrb}
\nc\klocint{\grk\fpowerser}
\nc\Klocfrac{K\laurentser}
\nc\tensor{\otimes} % algebraic tensor product

%\nc\Spec[1]{\mathrm{Spec} (#1)}
%\nc\Specf[1]{\mathrm{Specf} (#1)}

% TODO figure out optimal manner of handling infix macros
% something like \infixnewcommand{\T}{{bunewcommandhofstuff} {#1} {bunewcommandhofstuff} {#2} {bunewcommandhofstuff}}

% TODO read more about macros
% https://en.wikibooks.org/wiki/LaTeX/Macros

% relations, operations, unitary and binary symbols etc.
\nc\isom{\simeq} % TODO find optimal

\nc\ov[1]{/ {#1}}%'over' as in Scheme over k or galois extension over k

\nc\restr[2]{{% we make the whole thing an ordinary symbol
  \left.\kern-\nulldelimiterspace% automatically resize the bar with \right
  #1% the funewcommandtion
  \vphantom{\big|} % pretend it's a little taller at normal size
  \right|_{#2} % this is the delimiter
  }}

\nc\shf[1]{\mathcal{#1}}
\nc\idshf[1]{\mathcal{#1}}

\nc\projline{\mathbb{P}^1}

\nc\K{\mathbb{K}}

\nc\polyring[2]{R}

\nc\classgrp[1]{C_{#1}}
\nc\F{\mathbb{F}}
\nc\idealgenby[1]{\langle#1\rangle}
\nc\D{\mathbf{d}}
\nc\pt{\mathbf{pt}}
\nc\Gr{\textrm{Gr}} % affine grassmannian.

  \nc\Ga{\mathbb{G}_a}

  \nc\Gm{\mathbb{G}_m}

\nc\lpgrass{\mathcal{G}}
\nc\glpgrass{\mathcal{G}^P (I,Q)}

%, commonly denoted $\Gr_a$ or $G_{ K / {\ringint}$.}
% TODO = \mathbb{G}_{\mathcal{G}} = \mathcal{G}_{\mathcal{K}} \textrm{
% mod } \mathcal{G}_{\mathcal{O}}
% TODO structure nomenclature pre-amble by
% TODO 1. generality / importanewcommande
% TODO 2. chapter/lecture

% TODO group + order intro

%%%%%%%%%%%%%%%%%%%%%
%% classformations %%
%%
\nc\Flds{\Cat{Fields}}
\nc\cartdual{\mc{D}}

\nc\Mat[1]{\mathbf{Mat}_{#1\times#1}}
\nc\RH{\mathbb{\R}\textrm{H}}
\nc\tateshiftedby[1]{\left[#1\right]}
\nc\TateCat{\Cat{J_B}}
\nc\TateCatG{\Cat{J_G}}
\nc\B[1]{\mathbb{B} (#1)}
\nc\ptmod[1]{\mathbf{pt}\ov{#1}}
\nc\Ind{\mathrm{Ind}}
\nc\RHom{\mathrm{RHom}}
\nc\Hom{\mathrm{Hom}}
\nc\Map{\mathrm{Map}}
\nc\coH{\mathbf{H}}
\nc\hcoH[3]{\widehat{\coH^{#1}#2,#3}}
\nc\Ab{\Cat{Ab}} % TODO fix
\nc\Ker{\mathbf{Ker}}
\nc\Coker{\mathbf{Coker}}
\nc\dg{\textrm{DG}}
\rc\Vec{\Cat{Vec}}
\nc\VecF{\Cat{Vec}_{\grk}}
\nc\Coh{\Cat{Coh}}

\nc\Coind{\mathbf{Coind}}
\rc\Ind{\mathbf{Ind}}
\nc\Res{\mathbf{Res}}

%%%%%%%%%%%%%%%%%%%%%
%% classformations %%
%%
\nc\Pic{\mathbf{Pic}}
\nc\AJ{\mathbf{AJ}}
\nc\divisor[1]{\mathbf{#1}}
\nc\divisoridshf[1]{\idshf{I_{\divisor{#1}}}}
\nc\IndProSch{\Cat{IndProSch}}
\nc\FinIndProSch{\Cat{FinIndProSch}}
\nc\Pro{\mathbf{Pro}}
\nc\IndPro{\Ind-\Pro}
\nc\FinIndSch{\Cat{FinIndProSch}}
\nc\A{\Cat{CommIndSch_{\grk}}}

\nc\Fun[2]{#1 (#2)}
\nc\Frac[1]{\mathbf{Frac} (#1)}
\nc\curriedleftarg{\ldots}
\nc\Zp{\Z_p}

\nc\Qp{\mathbb{Q}_p}

% TODO group + order lec 3
% TODO review cardinality issues / small categories etc.
\nc\Set{\Cat{Set}} % Category of Sets

\nc\Hilb{\mathcal{H}\mathbb{ilb}}

\nc\disj{\sqcup} % TODO distinguish bigsqcup and sqcup
\nc\bij{\leftrightarrow}
\nc\sur{\twoheadedrightarrow}
\nc\inj{\rightarrowtail}
\nc\injects{\inj}
% \nc\xinj{\xrightarrowtail} % see sty-glg/commands package
% \nc\xhkinj{\xrightarrowtail} % see sty-glg/commands package
% \begin{document}
%   \[ f : G \xrightarrowtail[{\star}]{\text{\textbf{Grp}}} H \]
% \end{document}

% \xhookrightarrow from mathtools
% \[
% A\xhookrightarrow{} B\qquad A\xhookrightarrow{f\cirenewcommand g} B\qquad
% A \xhookrightarrow[(f\cirenewcommand g)\cirenewcommand h]{} B
% \]
% \xrightarrow from amsmath
% \[
% A\rightarrow{} B\qquad A\xrightarrow{f\cirenewcommand g} B\qquad
% A \xrightarrow[(f\cirenewcommand g)\cirenewcommand h]{} B
% \]

% TODO group, order + split off cohomology.tex

\nc\fst{\ensuremath{2^{\text{\tiny{st}}}}}
\nc\snd{\ensuremath{3^{\text{\tiny{nd}}}}}
\nc\thrd{\ensuremath{3^{\text{\tiny{rd}}}}}
\nc\nth{\ensuremath{n^{\text{\tiny{th}}}}}
% TODO find elegant resolution; possibilities:
% * \(n\)th
% * \usepackage{nth} NB: \nth{3} is different from $\nth{3}$
% * $n$-th
% * $n^{\text{\tiny th}}$
\nc\shfcoH[3]{\mathbf{H}^{#3} (#1,#2)}
\nc\hshfcoH[3]{\widehat{\shfcoH{#1,#2,#3}}}
\nc\Obj{\mathbf{Obj}}
% TODO nomencl Obj
\nc\grpring[2]{#1 \left[ #2 \right]}

\nc\Ext{\mathbf{Ext}}
\nc\Exts[3]{\Ext_{#1}^*\left(#2,#3 \right)}
% TODO nomencl Exts
\nc\dualmod[1]{#1^{*}}
% TODO investigate prefix macros e.g. {A}\Mod
\nc\LModCat[1]{#1-\Cat{LMod}}
\nc\RModCat[1]{#1-\Cat{RMod}}
\nc\indlim{\mathrm{lim}}

% TODO group + order indobjs.tex

\nc\Schemes{\Cat{Schemes}}

\nc\NoethSch{\Cat{NoethSch}}
\nc\NoethIntSch{\Cat{NoethIntSch}}

\nc\Sch[1]{\Cat{Sch}_{#1}}
\nc\IndSch[1]{\Cat{IndSch}_{#1}}

\nc\ModCat[1]{#1-\Cat{Mod}}

%\rc\dim[1]{\ensuremath{\mathbf{dim} (#1)}}

%%%%%%%%%%%%%%%%%%%%%%%%%%
%%
%%
% TODO learn how to typeset arrows in categories
% functors / presheaves / sheaves / group schemes etc.

\nc\grkAlg{\Cat{\grk-Algebras}}
\nc\ZAlg{\Cat{\Z-Algebras}}
%\nc\SL[1]{\mathbf{SL}_{#1}}

%{\ifstrequal{#1}{}{%if no scale provided
%  \SpecialLinearAbr}{


%% Index into Symbols and Notation
%% We collect all notation utilized throughouth this lecture series
%% we the hope their collation facillitates future functional/aesthetic changes

%%%%%%%%%%%%%%%%%%%%%%%%%%%%%%%%%%%%
%% TODO
%% - clarify class formation group k
%%    * little/big, more readable
%%    * visually distinguish notation for C_L / element
%% - Generate a glossary
%% - mark up structure for nomenclature / glossary inline
%%

% TODO fit these in
\nc\cc
{\ensuremath{\mathbf{CC}}} % TODO fix this
\nc\Bun{\ensuremath{\mathrm{Bun}}}

\nc\Spec[1]{\mathrm{Spec} (#1)}
\nc\Specf[1]{\mathrm{Specf} (#1)}

\nc\mK{\mathbf{K}} % TODO abstract out to notation.tex
\nc\Bil{\mathbf{Bil}} % TODO abstract to notation.tex
\nc\IndNSt{\Cat{Indn-Stack}}
% TODO figure out optimal manner of handling infix macros
% something like \infixnewcommand{\T}{{bunewcommandhofstuff} {#1} {bunewcommandhofstuff} {#2} {bunewcommandhofstuff}}

% TODO read more about macros
% https://en.wikibooks.org/wiki/LaTeX/Macros

% relations, operations, unitary and binary symbols etc.

%%%%%%%%%%%%%%%%%%%%%
%% classformations %%
%%
\rc\Pic{\mathbf{Pic}}
\rc\AJ{\mathbf{AJ}}
\rc\divisor[1]{\mathbf{#1}}
\rc\divisoridshf[1]{\idshf{I_{\divisor{#1}}}}
\rc\IndProSch{\Cat{IndProSch}}
\rc\FinIndProSch{\Cat{FinIndProSch}}
\rc\Pro{\mathbf{Pro}}
\rc\IndPro{\Ind-\Pro}
\rc\FinIndSch{\Cat{FinIndProSch}}
\rc\A{\Cat{CommIndSch_{\grk}}}

\rc\Fun[2]{#1( #2 )}
\rc\Frac[1]{\mathbf{Frac}( #1 )}
\rc\curriedleftarg{\ldots}
\rc\Zp{\Z_p}

%also called the formal completion at the point $p$ in the curve $\Spec{\Z}$}
\rc\Qp{\mathbb{Q}_p}

\rc\Hilb{\mathcal{H}\mathbb{ilb}}

% \rc\bij{\leftrightarrow}
% \rc\sur{\twoheadedrightarrow}
\rc\inj{\rightarrowtail}
\rc\injects{\inj}
% \rc\xinj{\xrightarrowtail} % see sty-glg/commands package
% \rc\xhkinj{\xrightarrowtail} % see sty-glg/commands package
% \begin{document}
%   \[ f : G \xrightarrowtail[{\star}]{\text{\textbf{Grp}}} H \]
% \end{document}

% \xhookrightarrow from mathtools
% \[
% A\xhookrightarrow{} B\qquad A\xhookrightarrow{f\cirenewcommand g} B\qquad
% A \xhookrightarrow[(f\cirenewcommand g)\cirenewcommand h]{} B
% \]
% \xrightarrow from amsmath
% \[
% A\rightarrow{} B\qquad A\xrightarrow{f\cirenewcommand g} B\qquad
% A \xrightarrow[(f\cirenewcommand g)\cirenewcommand h]{} B
% \]

% TODO group, order + split off cohomology.tex

\rc\fst{\ensuremath{2^{\text{\tiny{st}}}}}
\rc\snd{\ensuremath{3^{\text{\tiny{nd}}}}}
\rc\thrd{\ensuremath{3^{\text{\tiny{rd}}}}}
\rc\nth{\ensuremath{n^{\text{\tiny{th}}}}}
% TODO find elegant resolution; possibilities:
% * \(n\)th
% * \usepackage{nth} NB: \nth{3} is different from $\nth{3}$
% * $n$-th
% * $n^{\text{\tiny th}}$
\rc \shfcoH[3]{\mathbf{H}^{#3}(#1,#2)}
\rc\hshfcoH[3]{\widehat{\shfcoH{#1,#2,#3}}}
\rc\Obj{\mathbf{Obj}}
% TODO nomencl Obj
\rc\grpring[2]{#1 \left[ #2 \right]}

\rc\Ext{\mathbf{Ext}}
\rc\Exts[3]{\Ext_{#1}^*\left(#2,#3 \right)}
% TODO nomencl Exts
\rc\dualmod[1]{#1^{*}}
% TODO investigate prefix macros e.g. {A}\Mod
\rc\LModCat[1]{#1-\Cat{LMod}}
\rc\RModCat[1]{#1-\Cat{RMod}}
\rc\indlim{\mathrm{lim}}

% TODO group + order indobjs.tex

\rc\Schemes{\Cat{Sch}}

\rc\NoethSch{\Cat{NoethSch}}
\rc\NoethIntSch{\Cat{NoethIntSch}}

\rc\Sch[1]{\Cat{Sch}_{#1}}

\rc\IndSch[1]{\Cat{IndSch}_{#1}}

\rc\ModCat[1]{#1-\Cat{Mod}}
%commutative ring $A$.}

\rc\dim[1]{\ensuremath{\mathbf{dim}( #1 )}}

% TODO learn how to typeset arrows in categories
% functors / presheaves / sheaves / group schemes etc.

\rc\grkAlg{\Cat{\grk-Algebras}}
\rc\ZAlg{\Cat{\Z-Algebras}}
\nc\SL{\mathbf{SL}}
\nc\GL{\mathbf{GL}}
\nc\DMod{\Cat{DMod}}
\nc\IndSt{\Cat{IndStacks}}
\rc\IndNSt{\Cat{Ind}-\textsc{N}-\Cat{Stacks}}

\nc\Curve{\Sigma}
\nc\IxCHilb{\Hilb_{\Curve \times I}}
\nc\HilbCxI{\Hilb_{\Curve \times I}}
\nc\locLb{\mc{L}_{I,Q}}
\nc\Locvbpos{\{V_p\}_{p \in P}}
%\nc\IxCHilb{\Hilb_{\Curve \times I}}
%\nc\locLb{\mc{L}_{I,Q}}

% wip-surfaces.tex
\nc\Tor{\mathbf{Tor}}
%\nc\Ext{\mathbf{Tor}}
\nc\W{\mc{W}}
\nc\T{\mbf{T}}
%\nc\P{\mbb{P}}
\nc\Hirz{\mbf{H}}
%\nc\ringint{\mc{O}}
\nc\Tot{\mbf{Tot}}

%#\rc\P{\ensuremath{\mathbb{P}^2}}
\rc\P{\mathbb{P}}
\nc\Pbd{l_{\infty}}
\nc\blP{\widetilde{\mathbb{P}^2 \ov{\Z_2}}}
\nc\cM{\widetilde{\mathbb{M}(r,c)}}
\nc\M{\mathbb{M}(r,n)}
\rc\T{\mathbf{T}}
\nc\ef{\frac{\varepsilon_1}{\varepsilon_2}}
\nc\Vleft{\V{\sqrt{\frac{\ef}{\ef - 1}}}} \nc\Vright{\V{\sqrt{(1 - \ef)}}}
\nc\Verma[2]{\mathbf{V}_{#1,#2}}
\nc\VermaL[2]{\mathbf{L}_{#1,#2}}
\nc\V[1]{\mathbb{V}_{#1}}

\nc\lie[1]{\mathfrak{#1]}}
\nc\so{\lie{so}}
\nc\su{\lie{su}}
\rc\sl{\lie{sl}}
\nc\gl{\lie{gl}}

% thesis.tex
\rc\P{\mathbb{P}}
\rc\sp{\mathfrak{sp}}
\nc\g{\lie{g}}

\author{Raeez Lorgat}
\email{root@raeez.com}
\urladdr{http://math.raeez.com}


\title{Some Calculations Concerning The Topological String}

\begin{document}

  %%%%%%%%%%%%%%%%
  %% Title etc. %%
  %%

  \begin{abstract}Some computations and remarks on topological string theory.\end{abstract}

  \maketitle
  \tableofcontents
  \mbox{}
  %\nomenclature{$\Klocfrac$}{The field
of laurent series valued in $K$; equivalently, the fraction field of
series valued in $k$; equivalently, the ring of integers of the
completed local field.}
\nomenclature{$\Z$}{The ring of integers.}
\nomenclature{$\Q$}{The rational number field}
\nomenclature{$\R$}{The real number field}
\nomenclature{$\C$}{The complex number field}
\nomenclature{$\Spec{A}, \Specf{A\fpowerser}$}{The respective Prime and formal spectra of the commutative rings $A$ and $A\fpowerser$}
\nomenclature{$I,\idshf{I}$}{For an ideal $I$ in a commutative ring $A$,
$\idshf{I}$ denotes the associated ideal sheaf on $\Spec{A}$}
\nomenclature{$\projline$}{The projective line as algebro-geometric object, for
example, as represented in the functor-of-points yoga by two copies of $\Z[x]$
glued along a common $\Z\laurentpoly$ }
\nomenclature{$\K$}{The abstract total field}
\nomenclature{$\polyring{R}{n}$}{The ring of $R$-valued polynomials in the
formal variables $z_1,\ldots,z_n$ valued in the ring $R$.}
% TODO clarify distinction between polynomials 'in' and 'over' the $z_i$.
% what is the same: choose a convention for the symmetric algebra and its dual
\nomenclature{$ \FnS $}{Function field of a curve $\Sigma$ defined over $\grk$ }
\nomenclature{$\classgrp{\grk}$}{The class formation group associated to $\grk$.}
\nomenclature{$\F_p$}{A finite field of $p$ elements; for example the
quotient ring $\Z \ov{\idealgenby{p}}$}
\nomenclature{$\D = \D_a$}{The formal disc defined over $\grk$ with the
distinguished $\pt = a$ labeling the origin.}
\nomenclature{$\Gr$ }{The affine grassmannian of $G$.}
\nomenclature{$\Ga$}{The additive group as commutative reductive algebraic group}
\nomenclature{$\Gm$}{The multiplicative group as commutative reductive algebraic group.}
\nomenclature{$\lpgrass$}{The classical loop grassmannian of $G$.}
\nomenclature{$\glpgrass$}{The \textit{generalized} loop grassmannian of $G$}
\nomenclature{$\Flds$}{The Category of Fields}
\nomenclature{$\cartdual$}{Cartier duality on a category } % TODO say more}
\nomenclature{$\Mat{n}\grk$}{$n\times n$ matrices with coefficients in $\grk$}
\nomenclature{$\RH^*$}{Right derived functor of homology}
\nomenclature{$X\tateshiftedby{n}$}{The nth Tate Shift}
\nomenclature{$\TateCat$}{The tate category associated to a group $H$, defined as the
categorical quotient $\mathcal{A}^G_{N_G(\mc{A})}$} %N_G(\mathcal{A}
\nomenclature{$\B{G}$}{everyone's favourite quotient category}
\nomenclature{$\ptmod{G}$}{everyone's favourite quotient category}
%\nomenclature{$\hcoH{G,A,n}$}{blah}
\nomenclature{$\Vec_{\grk}$}{Category of Vector Spaces over $\grk$}
\nomenclature{$\divisor{p},\divisoridshf{\divisor{p}}$}{A divisor on some ambient space
along with its ideal sheaf}
\nomenclature{$\AJ$}{The Abel Jacobi map taking a divisor $\divisor{p}$
  supported on a space $X$ to $\AJ: \divisor{p} \mapsto \ringint_{X}
(-\divisor{p})\divisoridshf{I_{\divisor{p}}}$}
\nomenclature{$\IndProSch$}{The category of $\Ind$-$\Pro$-Schemes}
\nomenclature{$\FinIndProSch$}{The category of \textit{finite} $\IndPro$-Schemes}
\nomenclature{$\FinIndSch$}{The category of \textit{finite} $\Ind$-Schemes}
\nomenclature{$\A$}{The category of commutative indschemes over a ring $\grk$}
\nomenclature{$\Zp$}{The inverse limit of the inverse system of rings $\Z \ov{p^n \Z}$,
also called the formal completion at the point $p$ in the curve $\Spec{\Z}$}
\nomenclature{$\Set$}{The (small) category of Sets}
\nomenclature{$\Set$}{The (small) category of Sets}
\nomenclature{$\Hilb_{X}$}{The hilbert scheme of $X$ representing the moduli of finite length subschemes of $X$.}
%\nomenclature{$\shfcoH{X,\shf{I},n}$}{The $n$th cohomology of the sheaf $\shf{I}$ over the space $X$}.
\nomenclature{$\grpring{Z}{G}$}{The group ring of $G$}
\nomenclature{$\Bun G$}{The moduli space of $G$ bundles.}
\nomenclature[coh]{$\dualmod{M}$}{The dual module $\Hom{A}(M,A)$ associated to an object $M \in \ModCat{A}$ for some ring $A$}
\nomenclature{$\LModCat{A}$}{The category of left modules over a ring $A$.}
\nomenclature{$\RModCat{A}$}{The category of right modules over a ring $A$.}
\nomenclature{$\Schemes$}{The category of Schemes}
\nomenclature{$\NoethSch$}{The category of Noetherian Schemes}
\nomenclature{$\NoethIntSch$}{The category of Integral Noetherian Schemes}
\nomenclature{$\Sch{\grk}$}{The category of Schemes defined over $\grk$}
\nomenclature{$\IndSch{\grk}$}{The category of IndSchemes defined over $\grk$}
\nomenclature{$\ModCat{A}$}{The category of modules over a ring $A$.}
\nomenclature{$\SL_2$}{The reductive affine algebraic group scheme
  associated to the Special Linear Group of invertible determinant $1$ matrices
  i.e.\ the fiber $det^{-1}(-1)$ in $\Mat{2}$. Equivalently characterized via a
  \textit{functor of points} formalism e.g.\ $\grkAlg \rightarrow \Set$
represented in $\ZAlg$ by $\grpring{\Z}{\SL_2} \isom$ }
  %\polyring{Z}{4}\ov{\idealgenby{z_1z_3 - z_2z_4 - 1}}$}
\nomenclature{$G$}{A reductive group.}
\nomenclature{$\ringint$}{The local ring $\klocint$ ring of formal power series.}
% TODO frame as localization / indsystem
\nomenclature{$\grk$}{The ground field,
  often the complex numbers $\C$, and
  almost always embedded in the total field
  $K$.}
%\nomenclature[\ageom]{$\Spec{A}, \Specf{A\fpowerser}$}{The respective Prime and formal spectra of the commutative rings $A$ and $A\fpowerser$}
%\nomenclature{$I,\idshf{I}$}{For an ideal $I$ in a commutative ring $A$,
%$\idshf{I}$ denotes the associated ideal sheaf on $\Spec{A}$}
%\nomenclature{$\projline$}{The projective line as algebro-geometric object, for
%example, as represented in the functor-of-points yoga by two copies of $\Z[x]$
%glued along a common $\Z\laurentpoly$ }
%\nomenclature[\seccfm]{$\K$}{The abstract total field}
%\nomenclature[\ageom]{$\polyring{R}{n}$}{The ring of $R$-valued polynomials in the
%formal variables $z_1,\ldots,z_n$ valued in the ring $R$.}
%\nomenclature{$G$}{A reductive group.}
% algebraic fields, formal power series and their fraction fields
%\nomenclature[\ageom]{$\ringint$}{The local ring $\klocint$ ring of formal power
%series valued in $k$; equivalently, the ring of integers of the
%completed local field.}
%\nomenclature[\ageom]{$\grk$}{The ground field,
%  often the complex numbers $\C$, and
%  almost always embedded in the total field
%  $K$.}
%\nomenclature[\ageom]{$\Klocfrac$}{The field
%  of laurent series valued in $K$; equivalently, the fraction field of
%$\ringint$.}
%\nomenclature{$\Z$}{The ring of integers}
%\nomenclature{$\Q$}{The rational number field}
%\nomenclature{$\R$}{The real number field}
%\nomenclature{$\C$}{The complex number field}
%\nomenclature{$ \FnS $}{Function field of a curve $\Sigma$ defined over $\grk$ }
%\nomenclature{$\classgrp{\grk}$}{The class formation group associated to $\grk$.}
%\nomenclature{$\F_p$}{A finite field of $p$ elements; for example the
%\nomenclature{$\D = \D_a$}{The formal disc defined over $\grk$ with the
%distinguished $\pt = a$ labeling the origin.}
%\nomenclature{$\Gr$ }{The affine grassmannian of $G$.}
%\nomenclature{$\Ga$}{The additive group as commutative reductive algebraic group}
%\nomenclature{$\Set$}{The (small) category of Sets}
%\nomenclature{$\Qp$}{The fraction field of the ring of $p$-adic integers, i.e. $\Frac{\Zp}$}
%\nomenclature{$\Zp$}{The inverse limit of the inverse system of rings $\Z \ov{p^n \Z}$,
%\nomenclature[\seccfm]{$\A$}{The category of commutative indschemes over a ring $\grk$}
%\nomenclature[\seccfm]{$\FinIndSch$}{The category of \textit{finite} $\Ind$-Schemes}
%\nomenclature[\seccfm]{$\FinIndProSch$}{The category of \textit{finite} $\IndPro$-Schemes}
%\nomenclature{$\Hilb_{X}$}{The hilbert scheme of $X$ representing the moduli of finite length subschemes of $X$.}
%%\nomenclature{$\shfcoH{X,\shf{I},n}$}{The $n$th cohomology of the sheaf $\shf{I}$ over the space $X$}.
%\nomenclature[\seccfm]{$\Vec_{\grk}$}{Category of Vector Spaces over $\grk$}
%\nomenclature[\seccfm]{$\divisor{p},\divisoridshf{\divisor{p}}$}{A divisor on some ambient space
%along with its ideal sheaf}
%\nomenclature[\seccfm]{$\AJ$}{The Abel Jacobi map taking a divisor $\divisor{p}$
%  supported on a space $X$ to $\AJ: \divisor{p} \mapsto \ringint_{X}
%(-\divisor{p})\divisoridshf{I_{\divisor{p}}}$}
%\nomenclature[\seccfm]{$\IndProSch$}{The category of $\Ind$-$\Pro$-Schemes}
%\nomenclature[\seccoh]{$\grpring{Z}{G}$}{The group ring of $G$}
%\nomenclature[\seccoh]{$\Bun G$}{The moduli space of $G$ bundles.}
%\nomenclature[coh]{$\dualmod{M}$}{The dual module $\Hom{A}(M,A)$ associated to an object $M \in \ModCat{A}$ for some ring $A$}
%\nomenclature{$\LModCat{A}$}{The category of left modules over a ring $A$.}
%\nomenclature{$\RModCat{A}$}{The category of right modules over a ring $A$.}
%\nomenclature[\secindobj]{$\Schemes$}{The category of Schemes}
%\nomenclature[\secgaugetheory]{$\SL_2$}{The reductive affine algebraic group scheme
%  associated to the Special Linear Group of invertible determinant $1$ matrices
%  i.e.\ the fiber $det^{-1}(-1)$ in $\Mat{2}$. Equivalently characterized via a
%  \textit{functor of points} formalism e.g.\ $\grkAlg \rightarrow \Set$
%  represented in $\ZAlg$ by $\grpring{\Z}{\SL_2} \isom
%  \polyring{Z}{4}\ov{\idealgenby{z_1z_3 - z_2z_4 - 1}}$}
%\nomenclature[\secindobj]{$\NoethSch$}{The category of Noetherian Schemes}
%\nomenclature[\secindobj]{$\NoethIntSch$}{The category of Integral Noetherian Schemes}
%\nomenclature[\secindobj]{$\Sch{\grk}$}{The category of Schemes defined over $\grk$}
%\nomenclature[\secindobj]{$\IndSch{\grk}$}{The category of IndSchemes defined over $\grk$}
%\nomenclature[\secindobj]{$\ModCat{A}$}{The abelian category of modules over a
%%%%%%%%%%%%%%%%%%%%%%%%%%
%% %\nomenclature[\secgaugetheory] %%
%%
%%
%%\nomenclature{$\hcoH{G,A,n}$}{blah}
%\nomenclature{$\Gm$}{The multiplicative group as commutative reductive algebraic group.}
%\nomenclature{$\lpgrass$}{The classical loop grassmannian of $G$.}
%\nomenclature{$\glpgrass$}{The \textit{generalized} loop grassmannian of $G$}
%\nomenclature[\seccfm]{$\Flds$}{The Category of Fields}
%\nomenclature[\seccfm]{$\cartdual$}{Cartier duality on a category } % TODO say more}
%\nomenclature{$\Mat{n}\grk$}{$n\times n$ matrices with coefficients in $\grk$}
%\nomenclature{$\RH^*$}{Right derived functor of homology}
%\nomenclature{$X\tateshiftedby{n}$}{The $n$th Tate Shift}
%\nomenclature{$\TateCat$}{The tate category associated to a finite
%  abelian group $H$, defined as the categorical quotient
%  $\mathcal{A}^G_{N_G(\mathcal{A}}$}
%\nomenclature{$\B{G}$}{everyone's favourite quotient category}
%\nomenclature{$\ptmod{G}$}{everyone's favourite quotient category}
\nomenclature{$\DMod (X)$}{The category $\DMod$ of $D$-modules defined on $X$}
\nomenclature{$\IndSt$}{The Category of $\Ind$-Stacks}
\nomenclature{$\IndNSt$}{The Category of $\Ind$-$N$-Stacks}

  %\printnomenclature

\section{Introduction}

\subsection{Dreams}
  One of the prevalent themes at play in both contemporary theoretical physics
  and mathematics is that of \textit{duality}. In particular, we will adopt a
  form of \textit{Koszul Duality} as a guiding light in our quest to understand
  some of the salient characteristics of \textbf{M-Theory}.\par

  \begin{rmk}
    This introduction serves a motivational purpose, and subsequently lacks in
    rigour; Mathematicians may prefer to skip ahead to later sections.
  \end{rmk}

  Let $X$ be a manifold of dimension $\dim X = n$. We will consider
  M-theory on $\R \times X$ in the presence of a stack of $N$ 1-dimensional brane supported
  on $\R \times p$, for some point $p \in X$.

  In this scenario one expects to be able to construct two algebras

  \begin{itemize}
    \item The algebra $\mc{A}_N$ of operators on the stack of $N$ branes supported
      on $\R \times p$
    \item the algebra $\mc{B}$ of local operators of the gravitational theory
      on $\R \times X$. Operator product expansions in the direction of $\R$ equips $\mc{B}$ with the
  structure of associative algebra.
  \end{itemize}

  Holography then leads us to expect
  \begin{conj}
    In the limit as $\lim_{N \rightarrow \infty}$, $\mc{A}_N$ is the Koszul
    dual of $\mc{B}$.
  \end{conj}

  \begin{rmk}
    One should be able to exploit Witten's proposal for holographic
    calculations of the OPE.
  \end{rmk}

  These notes will amount to a rigorous check of this form of holography in a
  few (relatively) simple examples.

\subsection{String Field Theory and Koszul Duality}
  We expect Koszul duality for string field theories to be captured at the
  level of Calabi-Yau categories and their respective cyclic homologies, via
  the theorem of Loday-Quillen-Tsygan, which says
  \begin{thm}
    \textbf{Loday-Quillen-Tsygan} the Chevalley-Eilenberg-Lie homology of the
    lie algebra of infinite matrices over a unital associative algebra $A$ is
    generated by the cyclic homology of $A$ as an exterior algebra.
  \end{thm}

  The associative algebra $A$ will coincide with the algebra of cochains arising in the
  large $N$ limit, dual to the cyclic homology produced on the gravitational
  side.\par

  Thus, for example, we will obtain an explicit matching between single string states and
  generators given by single trace operators\footnotemark\footnotetext{At the
  classical level} at the large $N$ limit, while OPE's will coincide with
  string scattering amplitudes.

\subsection{Flavours of Topological Strings}
Topological strings manifest in various flavours. Common examples include

\begin{itemize}
  \item The \textbf{A-model} where $X$ is symplectic and branes are
    lagrangians $L \inj X$
  \item The \textbf{B-model} where $X$ is calabi-yau and branes are coherent
    sheaves.
  \item \textbf{Mixed A-B models} e.g.\ on $X \times X'$ with branes given by
    $L\times Z$ for $L \inj X$ lagrangian and $Z \inj Y$ holomorphic.
\end{itemize}

\begin{rmk}
  The above examples do not exhaust the full landscape of topological
  strings. % TODO give example
\end{rmk}

\begin{rmk}
  The A-model is only interesting with world-sheet instantons taken into
  account.
\end{rmk}

Regardless of the context we will work in, we'll need to understand

\begin{itemize}
  \item The open string theory, which is the theory on a Brane, and
  \item The closed string field theory, of a gravitational nature
\end{itemize}

\subsection{Open Strings}
If $\mc{C}$ is a Calabi-Yau category, with $\mc{F} \in \Cat{Ob}\mc{C}$, one
can construct a field theory with a space of
fields\footnotemark\footnotetext{If we just produce a category, then the
  resulting field theory would be defined on a point. If instead we produce a
  sheaf of categories, we get a field theory on the support of this sheaf of
categories i.e.\ a field theory on the given brane}given by
$$\RHom(\mc{F},\mc{F})[1],$$
also known as the open string states. The action for this field theory is
built in terms of the $A_{\infty}$ structure\footnotemark\footnotetext{Via a
standard procedure going back to Witten} on this derived $\Hom$ space.

  \section{First Example}

\subsection{Our setup}
  Let $X$ be $\R^2_A \times \C^2_B$. Branes will be of the form $L \times Z$
  where $L$ is a (lagrangian) line in $\R^2_A$ and $Z$ is the support of a
  coherent sheaf on $\C^2_B$. Consider, in particular, the Brane $$\R \times p
  \inj \R^2_A \times \C^2_B$$ for some point $p \in \C^2_B$.

\subsection{Open String Field States}
  The open string states will be a tensor product of A-model and B-model
  states. If one consults the mathematical literature one would find the
  following prescription:

  \begin{itemize}
    \item $\R \inj \R^2_A$ has open string states given by the Floer
      cohomology, which in this instance is uninteresting and reduces to the
      regular cohomology of $\R$. Since we need this in its sheaf incarnation,
      our open sring states will be the de Rahm complex $\Omega^*(\R)$
    \item $\{p\} \inj \C^2_B$ has open string states is given by
      $\Ext^*_{\mc{O}(\C)}(\C,\C)$ which has the form of an exterior algebra on
      2 generators $ = \C[\varepsilon_1,\varepsilon_2]$, each generator
      representing a normal direction.
  \end{itemize}

\subsection{Open String Field Theory}

We hence find that the fields\footnotemark\footnotetext{in the
Batalin-Vilkovisky formalism} are given by
$$\alpha \in \Omega^*(\R)[\varepsilon_1,\varepsilon_2] \otimes
\lie{gl}_N[1]$$ with action $$S(\alpha) = \int_{\R \times
\C^{0|2}}\frac{1}{2}Tr(\alpha d\alpha) + \frac{1}{3} Tr \alpha^3$$

\begin{rmk}
  Note that in the action we perform a Berezin integral over $\C^{0|2}$
\end{rmk}
\begin{rmk}
  The shift of 1 in the space of fields above means that the $1$-forms are
  now in ghost number 0, i.e.\ is a bosonic gauge field. Similarly, the
  scalars $\varepsilon_i$ (representing the motions of brane) are also ghost
  number 0.
\end{rmk}

\subsubsection{De-BV-fying the open string field theory}
  % reference Alberto's writeup
  In the Non-BV setup, i.e.\ for a starting point from which we can follow
  the procedure outlined as in ?? to derive the above, we write down ghosts,
  the anti-fields, the anti-fields to the ghosts etc. to find a Quantum-Mechanical system with fields
  $$A \in \Omega^1(\R)\otimes \lie{gl}_N$$ a connection form and two scalars
  $$\phi_1,\phi_2 \in \Omega^0(\R)\otimes\lie{gl}_N$$ and action $$S = \int
  Tr(\phi_1, d_A\phi_2)$$ with the usual gauge symmetry.

\begin{rmk}
  We can also proceed directly within the CY category. There we take a given
  object, tensor it with $\C^N$; passage to endomorphisms yields
  endomorphisms tensored with that of $\C^N$. More generally, one should take
  proceed via the cyclic cohomology of the category. What we really need is
  to be able to consider is an integral over world sheets decorated with
  cyclic cohomology classes. Then the 3-point function should be computable
  via the cup-product in cyclic cohomology together with the trace. However,
  the higher $n$-point functions are trickier.
\end{rmk}

\begin{rmk}
  This is literally 1-dimensional chern simons theory (via AKSZ formalism).
\end{rmk}

\begin{rmk}
  we can also realize this via maps to $\B{G}$ from $\R\times \C^{0|2}$; this
  point of view makes manifest the origin of this theory as a reduction of chern simons on a
2-torus. (The two odd directions generate the cohomology of the 2-torus.)
Elaborating on this point of view with an analysis of T-duality recovers
Witten's prescriptions for Chern Simons theory via topological string theory.
\end{rmk}

\subsection{Closed String States}
Similarly, the closed string state space will be of the form
$$\mathrm{A-model closed states}\otimes \mathrm{B-model closed
states}$$
For the A-model on a symplectic manifold, as usual, we get the cohomology of
that manifold; but again as we are constructing a field tehory on $\R^2$ we
take the de Rahm cohomology complex.
For the B-model, following BCOV's analysis on a CY 3-fold, we consider $$\Ker \partial \subset
\mbf{PV}^{*,*}(\C^2) \simeq
\Omega^{0,*}(\C^2)[\partial_{z_1},\partial_{z_2}]$$
i.e.\ the kernel of a certain operator on the space of poly-vector fields.
\begin{rmk}
  The $\partial_{z_i}$ are odd.
\end{rmk}

\begin{rmk}
  The constructions here are much more homologically involved. Updates coming soon.
\end{rmk}

Observe that the action of $\partial$ on PolyVector fields $$\partial :
\mbf{PV}^{i,*} \rightarrow \mbf{PV}^{i-1,*}$$
and that $$\mbf{PV}^{0,*}\subset \Ker\partial.$$ Now since $\C^2$ is symplectic, we
find$$\mbf{PV}^{1,*}(\C^2) \simeq
\Omega^{1,*}(\C^2)$$ with the restriction of the operator $\partial$ acting as the holomorphic de
Rahm operator. Hence we are considering the complex of
closed\footnotemark\footnotetext{closed in the holomorphic sense} holomorphic
1-forms.

\begin{rmk}
  Following BCOV, we have stipulated that the fields must satisfy a
  differential equation, which inevitably produces a technical nightmare.
  Further, what is worse is that the action functional (coinciding with the
  inverse of $\partial$) is non-local.
\end{rmk}

Also, note that when restricting to poly-2-vector-fields $$\mbf{PV}^{2,*}(\C^2) \supset \Ker\partial$$
$\Ker\partial$ picks out poly-2-vector-fields that are independent of $z$,
hence
$$\mbf{PV}^{2,*}(\C^2) \supset \Ker\partial \simeq \C$$
(the cohomology is $\C$), which is again (almost) trivial. What is more, by the
definition of a propagator in BCOV theory, this does not propagate---hence form
background fields which we'll ignore.

\begin{rmk}
  this statement generalizes the sheafy setting.
\end{rmk}

\subsection{Closed String Field Theory}

Combining this analysis, for our candidate for the closed string field
theory we are led to divergence free vector fields; equivalently, in this case, symplectic
vector fields. Now we opt to replace $\Ker\partial$ by the image
of $\partial$ i.e.\ by $$\Omega^{0,*} - \mathrm{Hamiltonian Vector
Fields}$$

\begin{rmk}
  We have replaced one complex by another that is almost---but not
  quite---quasi-isomorphic: they differ by the cohomology of $\C$, which is
  (almost) trivial.
\end{rmk}

\subsubsection{The Action Functional}

We can now write down the action, for $$\alpha \in
\Omega^*(\R^2)\hat{\otimes}\Omega^{0,*}(\C^2)[1]$$
$$\beta \in \Omega^*(\R^2)\hat{\otimes}\Omega^{0,*}(\C^2)[2]$$
the action is an integral over real 6-dimensional space, cubic in the fields
$$\int_{\R^2 \times \C^2} \beta\bar{\partial}\alpha dz_1 dz_2 +
\int_{\R^2\times \C^2} \beta \partial \alpha \partial \alpha$$

\begin{rmk}
  Observe that the second term of the action can be expressed as
  $$\int_{\R^2\times \C^2} \beta \partial \alpha \partial \alpha =
  \int_{\R^2\times \C^2} \beta\{\alpha,\alpha\}dz_1 dz_2$$
  where the expression $\partial \alpha \partial \alpha$ can be expressed as a
  poisson bracket, hence we see the lie bracket on hamiltonian vector fields
  entering the action functional.
\end{rmk}

\begin{rmk}
  Before, the quantum mechanical system lived on a line (extent of the brane), while this gravitational theory occupies all of space-time.
\end{rmk}

\subsubsection{De-BV-fying the closed string field theory}

\begin{rmk}
  This theory is closely related to a 4-dimensional BF theory.
  TODO more
\end{rmk}

Consider space-time $$\R^2_{x_i}\times\C^2_{z_i}$$ where the subscripts
indicate coordinates. Paying attention to ghost number zero components
$$\beta = \beta_{x_1 x_2} dx_1 dx_2 + \beta_{x_i \bar{z}_j} dx_i d\bar{z}_j +
\beta_{\bar{z}_1\bar{z}_2 d\bar{z}_1d\bar{z}_2}$$
$$\alpha = \alpha_{x_i}dx_i + \alpha_{\bar{z}_j}d\bar{z}_j$$
and considering the action as given above, we see that $\beta$ has $1$-form gauge
symmetry, while $\alpha$ has $0$-form gauge symmetry.
\begin{rmk}
  As in the previously alluded to $4$-dimensional BF theory, $\beta$---having
  $1$-form gauge symmetry---also has secondary gauge symmetry, which is why we
  find a large complex. TODO more.
\end{rmk}

Analysing $\alpha$, as a Hamiltonian vector field on $\C^2$, we see the
term $\alpha_{\bar{z}_j}d\bar{z}_j$ is the Beltrami differential, where we
deform $\C^2$ as a symplectic surface. The term $\alpha_{x_i}dx_i$ is a
connection on $\R^2$ valued in Hamiltonian vector fields; the terms together
imply that we have a flat bundle of $\C^2$'s over $\R^2$. In fact, the
equations of motion determine an integrable deformation of complex structure.
Analysing $\beta$ in the action functional, it only appears linearly. This
implies they're just lagrangian multiplier fields in enforcing the equations
resulting in the integrable deformation of complex structure + flatness of the
bundle of $\C^2$'s over $\R^2$.
\begin{rmk}
  This final statement is analogous to BF theory where $B$ is a lagrangian
  multiplier and $F$ is zero, hence why we see in the next section that the
  phase space is a cotangent bundle.
\end{rmk}

\subsubsection{Phase Space}
If we put this on $$\R \times S^1 \times X$$ where $X$ is homolomorphic
symplectic, then the phase space is $$T^*(\mathrm{Moduli of
holomorphic-symplectic surfaces fibered over} S^1 + \mathrm{flat
connection})$$
\begin{rmk}
  If we work in perturbation theory near $X$ we will find the above moduli
  space near this configuration.
\end{rmk}

\subsection{Closed String Fields}

Given a closed string field, we should obtain a deformation of the gauge theory
on the brane. For example, if the brane is on the line $x_1$ and the closed
string  field is $$dx_1 z_1^k z_2^l$$ we ought to obtain a deformation of our
quantum mechanical system on the line $x_1$. We can identify this deformation
by deforming the action to $$\int Tr \phi_1 d_A \phi_2 + \int dx_1 Tr \phi_1^k
\phi_2^l$$ by adding on a Hamiltonian. Many matrix models of hamiltonians can
be engineered by these backgrounds.

\begin{rmk}
  Geometrically, this just amounts to saying that the $\C^2$ has a connection,
  allowing parallel transport.
\end{rmk}

\subsection{Analysis of $N \rightarrow \infty$}

\end{document}
