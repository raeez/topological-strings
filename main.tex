\documentclass[12pt]{amsart}

\input{preamble.tex}
\input{mathmacros.tex}
\author{Raeez Lorgat}
\email{root@raeez.com}
\urladdr{http://math.raeez.com}


\title{Notes on the Topological Super-String}

\begin{document}

  %%%%%%%%%%%%%%%%
  %% Title etc. %%
  %%

  \begin{abstract}Some computations and remarks on Kodaira-Spencer Gravity and
  quantum string amplitudes in topological string theory. Much of the analysis has its origin in the work of
Bershadsky-Cecotti-Ooguri-Vafa.\end{abstract}

  \maketitle
  \tableofcontents
  \mbox{}
  %\input{nomenclature.tex}
  %\printnomenclature

  \section{Introduction}

  \subsection{Dreams}
  One of the prevalent themes at play in both contemporary theoretical physics
  and mathematics is that of \textit{duality}. In particular, we will adopt a
  form of \textit{Koszul Duality} as a guiding light in our quest to understand
  some of the salient characteristics of \textbf{M-Theory}.\par

  \begin{rmk}
    This introduction serves a motivational purpose, and subsequently lacks in
    rigour; Mathematicians may prefer to skip ahead to later sections.
  \end{rmk}

  Let $X$ be a manifold of dimension $\dim X = n$. We will consider
  M-theory on $\R \times X$ in the presence of a stack of $N$ 1-dimensional brane supported
  on $\R \times p$, for some point $p \in X$.

  In this scenario one expects to be able to construct two algebras

  \begin{itemize}
    \item The algebra $\mc{A}_N$ of operators on the stack of $N$ branes supported
      on $\R \times p$
    \item the algebra $\mc{B}$ of local operators of the gravitational theory
      on $\R \times X$. Operator product expansions in the direction of $\R$ equips $\mc{B}$ with the
  structure of associative algebra.
  \end{itemize}

  Holography then leads us to expect
  \begin{conj}
    In the limit as $\lim_{N \rightarrow \infty}$, $\mc{A}_N$ is the Koszul
    dual of $\mc{B}$.
  \end{conj}

  \begin{rmk}
    One should be able to exploit Witten's proposal for holographic
    calculations of the OPE.
  \end{rmk}

  These notes will amount to a rigorous check of this form of holography in a
  few (relatively) simple examples.

  \subsection{String Field Theory and Koszul Duality}
  We expect Koszul duality for string field theories to be captured at the
  level of Calabi-Yau categories and their respective cyclic homologies, via
  the theorem of Loday-Quillen-Tsygan, which says
  \begin{thm}
    \textbf{Loday-Quillen-Tsygan} the Chevalley-Eilenberg-Lie homology of the
    lie algebra of infinite matrices over a unital associative algebra $A$ is
    generated by the cyclic homology of $A$ as an exterior algebra.
  \end{thm}

  The associative algebra $A$ will coincide with the algebra of cochains arising in the
  large $N$ limit, dual to the cyclic homology produced on the gravitational
  side.\par

  Thus, for example, we will obtain an explicit matching between single string states and
  generators given by single trace operators\footnotemark\footnotetext{At the
  classical level} at the large $N$ limit, while OPE's will coincide with
  string scattering amplitudes.

  \subsection{Flavours of Topological Strings}
  Topological strings manifest in various flavours. Common examples include

  \begin{itemize}
    \item The \textbf{A-model} where $X$ is symplectic and branes are
      lagrangians $L \inj X$
    \item The \textbf{B-model} where $X$ is calabi-yau and branes are coherent
      sheaves.
    \item \textbf{Mixed A-B models} e.g.\ on $X \times X'$ with branes given by
      $L\times Z$ for $L \inj X$ lagrangian and $Z \inj Y$ holomorphic.
  \end{itemize}

  \begin{rmk}
    The above examples do not exhaust the full landscape of topological
    strings. % TODO give example
  \end{rmk}

  \begin{rmk}
    The A-model is only interesting with world-sheet instantons taken into
    account.
  \end{rmk}

  Regardless of the context we will work in, we'll need to understand

  \begin{itemize}
    \item The open string theory, which is the theory on a Brane, and
    \item The closed string field theory, of a gravitational nature
  \end{itemize}

  \subsection{Open Strings}
  If $\mc{C}$ is a Calabi-Yau category, with $\mc{F} \in \Cat{Ob}\mc{C}$, one
  can construct a field theory with a space of
  fields\footnotemark\footnotetext{If we just produce a category, then the
    resulting field theory would be defined on a point. If instead we produce a
    sheaf of categories, we get a field theory on the support of this sheaf of
  categories i.e.\ a field theory on the given brane}given by
  $$\RHom(\mc{F},\mc{F})[1],$$
  also known as the open string states. The action for this field theory is
  built in terms of the $A_{\infty}$ structure\footnotemark\footnotetext{Via a
  standard procedure going back to Witten} on this derived $\Hom$ space.

  \section{First Example}

  \subsection{Our setup}
  Let $X$ be $\R^2_A \times \C^2_B$. Branes will be of the form $L \times Z$
  where $L$ is a (lagrangian) line in $\R^2_A$ and $Z$ is the support of a
  coherent sheaf on $\C^2_B$. Consider, in particular, the Brane $$\R \times p
  \inj \R^2_A \times \C^2_B$$ for some point $p \in \C^2_B$.

  \subsection{Open String Field States}
  The open string states will be a tensor product of A-model and B-model
  states. If one consults the mathematical literature one would find the
  following prescription:

  \begin{itemize}
    \item $\R \inj \R^2_A$ has open string states given by the Floer
      cohomology, which in this instance is uninteresting and reduces to the
      regular cohomology of $\R$. Since we need this in its sheaf incarnation,
      our open sring states will be the de Rahm complex $\Omega^*(\R)$
    \item $\{p\} \inj \C^2_B$ has open string states is given by
      $\Ext^*_{\mc{O}(\C)}(\C,\C)$ which has the form of an exterior algebra on
      2 generators $ = \C[\varepsilon_1,\varepsilon_2]$, each generator
      representing a normal direction.
  \end{itemize}

  \subsection{Open String Field Theory}

  We hence find that the fields\footnotemark\footnotetext{in the
  Batalin-Vilkovisky formalism} are given by
  $$\alpha \in \Omega^*(\R)[\varepsilon_1,\varepsilon_2] \otimes
  \lie{gl}_N[1]$$ with action $$S(\alpha) = \int_{\R \times
  \C^{0|2}}\frac{1}{2}Tr(\alpha d\alpha) + \frac{1}{3} Tr \alpha^3$$

  \begin{rmk}
    Note that in the action we perform a Berezin integral over $\C^{0|2}$
  \end{rmk}
  \begin{rmk}
    The shift of 1 in the space of fields above means that the $1$-forms are
    now in ghost number 0, i.e.\ is a bosonic gauge field. Similarly, the
    scalars $\varepsilon_i$ (representing the motions of brane) are also ghost
    number 0.
  \end{rmk}

  \begin{rmk}
    % reference Alberto's writeup
    In the Non-BV setup, i.e.\ for a starting point from which we can follow
    the procedure outlined as in ?? to derive the above, we write down ghosts,
    the anti-fields, the anti-fields to the ghosts etc. to find a Quantum-Mechanical system with fields
    $$A \in \Omega^1(\R)\otimes \lie{gl}_N$$ a connection form and two scalars
    $$\phi_1,\phi_2 \in \Omega^0(\R)\otimes\lie{gl}_N$$ and action $$S = \int
    Tr(\phi_1, d_A\phi_2)$$ with the usual gauge symmetry.
  \end{rmk}

  \begin{rmk}
    We can also proceed directly within the CY category. There we take a given
    object, tensor it with $\C^N$; passage to endomorphisms yields
    endomorphisms tensored with that of $\C^N$. More generally, one should take
    proceed via the cyclic cohomology of the category. What we really need is
    to be able to consider is an integral over world sheets decorated with
    cyclic cohomology classes. Then the 3-point function should be computable
    via the cup-product in cyclic cohomology together with the trace. However,
    the higher $n$-point functions are trickier.
  \end{rmk}

  \begin{rmk}
    This is literally 1-dimensional chern simons theory (via AKSZ formalism).
  \end{rmk}

  \begin{rmk}
    we can also realize this via maps to $\B{G}$ from $\R\times \C^{0|2}$; this
    point of view makes manifest the origin of this theory as a reduction of chern simons on a
  2-torus. (The two odd directions generate the cohomology of the 2-torus.)
  Elaborating on this point of view with an analysis of T-duality recovers
  Witten's prescriptions for Chern Simons theory via topological string theory.
  \end{rmk}

  \subsection{Closed String States}
  Similarly, the closed string state space will be of the form
  $$\mathrm{A-model closed states}\otimes \mathrm{B-model closed
  states}$$
  For the A-model on $\R^2$ we have the de Rahm cohomology.
  For the B-model, following BCOV we consider $$\Ker \partial \subset
  \mbf{PV}^{*,*}(\C^2) \simeq
  \Omega^{0,*}(\C^2)[\partial_{z_1},\partial_{z_2}]$$

  \begin{rmk}
    The $\partial_{z_i}$ are odd.
  \end{rmk}

  \begin{rmk}
    The constructions here are much more homologically involved. Updates coming soon.
  \end{rmk}

  Observe that the action of $\partial$ on PolyVector fields $$\partial :
  PV^{i,*} \rightarrow PV^{i-1,*}$$
  and that $$PV^{0,*}\subset \Ker\partial$$. Now since $\C^2$ is symplectic, we
  find$$PV^{1,*}(\C^2) \simeq
  \Omega^{1,*}(\C^2)$$ where the operator $\partial$ becomes the holomorphic de
  Rahm operator. Hence we are considering the complex of
  closed\footnotemark\footnotetext{closed in the holomorphic sense} holomorphic
  1-forms.

  \begin{rmk}
    Following BCOV, we have stipulated that the fields must satisfy a
    differential equation, which inevitably produces a technical nightmare.
    Further, what is worse is that the action functional (coinciding with the
    inverse of $\partial$) is non-local.
  \end{rmk}
  Thus we are led to divergence free vector fields; equivalently symplectic
  vector fields in this case. Now we opt to replace $\Ker\partial$ by the image
  of $\partial$ i.e.\ by $$\Omega^{0,*} - \mathrm{Hamiltonian Vector
  Fields}$$

  \begin{rmk}
    We have replaced one complex by another that is almost---but not
    quite---quasi-isomorphic: they differ by the cohomology of $\C$, which is
    (almost) trivial.
  \end{rmk}

  \subsection{Analysis of $N \rightarrow \infty$}

\end{document}
