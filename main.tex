\documentclass[12pt]{amsart}

\input{preamble.tex}
\input{mathmacros.tex}
\author{Raeez Lorgat}
\email{root@raeez.com}
\urladdr{http://math.raeez.com}


\title{Some Calculations Concerning The Topological String}

\begin{document}

  %%%%%%%%%%%%%%%%
  %% Title etc. %%
  %%

  \begin{abstract}Some computations and remarks on Kodaira-Spencer Gravity and
  quantum string amplitudes in topological string theory. Much of the analysis has its origin in the work of
Bershadsky-Cecotti-Ooguri-Vafa.\end{abstract}

  \maketitle
  \tableofcontents
  \mbox{}
  %\input{nomenclature.tex}
  %\printnomenclature

\section{Introduction}

\subsection{Dreams}
  One of the prevalent themes at play in both contemporary theoretical physics
  and mathematics is that of \textit{duality}. In particular, we will adopt a
  form of \textit{Koszul Duality} as a guiding light in our quest to understand
  some of the salient characteristics of \textbf{M-Theory}.\par

  \begin{rmk}
    This introduction serves a motivational purpose, and subsequently lacks in
    rigour; Mathematicians may prefer to skip ahead to later sections.
  \end{rmk}

  Let $X$ be a manifold of dimension $\dim X = n$. We will consider
  M-theory on $\R \times X$ in the presence of a stack of $N$ 1-dimensional
  branes supported
  on $\R \times p$, for some point $p \in X$.

  In this scenario one expects to be able to construct two algebras

  \begin{itemize}
    \item The algebra $\mc{A}_N$ of operators on the stack of $N$ branes supported
      on $\R \times p$
    \item the algebra $\mc{B}$ of local operators of the gravitational theory
      on $\R \times X$. Operator product expansions in the direction of $\R$ equips $\mc{B}$ with the
  structure of associative algebra.
  \end{itemize}

  Holography then leads us to expect
  \begin{conj}
    In the limit as $\lim_{N \rightarrow \infty}$, $\mc{A}_N$ is the Koszul
    dual of $\mc{B}$.
  \end{conj}

  \begin{rmk}
    One should be able to exploit Witten's proposal for holographic
    calculations of the OPE.
  \end{rmk}

  These notes will amount to a rigorous check of this form of holography in a
  few (relatively) simple examples.

\subsection{String Field Theory and Koszul Duality}
  We expect Koszul duality for string field theories to be captured at the
  level of Calabi-Yau categories and their respective cyclic homologies, via
  analogues of
  the theorem of Loday-Quillen-Tsygan, which says
  \begin{thm}
    \textbf{Loday-Quillen-Tsygan} the Chevalley-Eilenberg-Lie homology of the
    lie algebra of infinite matrices over a unital associative algebra $A$ is
    generated by the cyclic homology of $A$ as an exterior algebra.
  \end{thm}

  The associative algebra $A$ will coincide with the algebra of cochains arising in the
  large $N$ limit, dual to the cyclic homology produced on the gravitational
  side.\par

  Thus, for example, we will obtain an explicit matching between single string states and
  generators given by single trace operators\footnotemark\footnotetext{At the
  classical level} at the large $N$ limit, while OPE's will coincide with
  string scattering amplitudes.

\subsection{Flavours of Topological Strings}
Topological strings manifest in various flavours. Common examples include

\begin{itemize}
  \item The \textbf{A-model} where $X$ is symplectic and branes are
    lagrangians $L \inj X$
  \item The \textbf{B-model} where $X$ is calabi-yau and branes are coherent
    sheaves.
  \item \textbf{Mixed A-B models} e.g.\ on $X \times X'$ with branes given by
    $L\times Z$ for $L \inj X$ lagrangian and $Z \inj Y$ holomorphic.
\end{itemize}

\begin{rmk}
  The above examples do not exhaust the full landscape of topological
  strings. % TODO give example
\end{rmk}

\begin{rmk}
  The A-model is only interesting with world-sheet instantons taken into
  account.
\end{rmk}

Regardless of the context we will work in, we'll need to understand

\begin{itemize}
  \item The open string theory, which is the theory on a Brane, and
  \item The closed string field theory, of a gravitational nature
\end{itemize}

\subsection{Open Strings}
If $\mc{C}$ is a Calabi-Yau category, with $\mc{F} \in \Cat{Ob}\mc{C}$, one
can construct a field theory with a space of
fields\footnotemark\footnotetext{If we just produce a category, then the
  resulting field theory would be defined on a point. If instead we produce a
  sheaf of categories, we get a field theory on the support of this sheaf of
categories i.e.\ a field theory on the given brane}given by
$$\RHom(\mc{F},\mc{F})[1],$$
also known as the open string states. The action for this field theory is
built in terms of the $A_{\infty}$ structure\footnotemark\footnotetext{Via a
standard procedure going back to Witten} on this derived $\Hom$ space.

  \section{First Example}

\subsection{Our setup}
  Let $X$ be $\R^2_A \times \C^2_B$. Branes will be of the form $L \times Z$
  where $L$ is a (lagrangian) line in $\R^2_A$ and $Z$ is the support of a
  coherent sheaf on $\C^2_B$. Consider, in particular, the Brane $$\R \times p
  \inj \R^2_A \times \C^2_B$$ for some point $p \in \C^2_B$.

\subsection{Open String Field States}
  The open string states will be a tensor product of A-model and B-model
  states. If one consults the mathematical literature one would find the
  following prescription:

  \begin{itemize}
    \item $\R \inj \R^2_A$ has open string states given by the Floer
      cohomology, which in this instance is uninteresting and reduces to the
      regular cohomology of $\R$. Since we need this in its sheaf incarnation,
      our open sring states will be the de Rahm complex $\Omega^*(\R)$
    \item $\{p\} \inj \C^2_B$ has open string states is given by
      $\Ext^*_{\mc{O}(\C)}(\C,\C)$ which has the form of an exterior algebra on
      2 generators $ = \C[\varepsilon_1,\varepsilon_2]$, each generator
      representing a normal direction.
  \end{itemize}

\subsection{Open String Field Theory}

We hence find that the fields\footnotemark\footnotetext{in the
Batalin-Vilkovisky formalism} are given by
$$\alpha \in \Omega^*(\R)[\varepsilon_1,\varepsilon_2] \otimes
\lie{gl}_N[1]$$ with action $$S(\alpha) = \int_{\R \times
\C^{0|2}}\frac{1}{2}Tr(\alpha d\alpha) + \frac{1}{3} Tr \alpha^3$$

\begin{rmk}
  Note that in the action we perform a Berezin integral over $\C^{0|2}$
\end{rmk}
\begin{rmk}
  The shift of 1 in the space of fields above means that the $1$-forms are
  now in ghost number 0, i.e.\ is a bosonic gauge field. Similarly, the
  scalars $\varepsilon_i$ (representing the motions of brane) are also ghost
  number 0.
\end{rmk}

\subsubsection{De-BV-fying the open string field theory}
  % reference Alberto's writeup
  In the Non-BV setup, i.e.\ for a starting point from which we can follow
  the procedure outlined as in ?? to derive the above, we write down ghosts,
  the anti-fields, the anti-fields to the ghosts etc. to find a Quantum-Mechanical system with fields
  $$A \in \Omega^1(\R)\otimes \lie{gl}_N$$ a connection form and two scalars
  $$\phi_1,\phi_2 \in \Omega^0(\R)\otimes\lie{gl}_N$$ and action $$S = \int
  Tr(\phi_1, d_A\phi_2)$$ with the usual gauge symmetry.

\begin{rmk}
  We can also proceed directly within the CY category. There we take a given
  object, tensor it with $\C^N$; passage to endomorphisms yields
  endomorphisms tensored with that of $\C^N$. More generally, one should take
  proceed via the cyclic cohomology of the category. What we really need is
  to be able to consider is an integral over world sheets decorated with
  cyclic cohomology classes. Then the 3-point function should be computable
  via the cup-product in cyclic cohomology together with the trace. However,
  the higher $n$-point functions are trickier.
\end{rmk}

\begin{rmk}
  This is literally 1-dimensional chern simons theory (via AKSZ formalism).
\end{rmk}

\begin{rmk}
  we can also realize this via maps to $\B{G}$ from $\R\times \C^{0|2}$; this
  point of view makes manifest the origin of this theory as a reduction of chern simons on a
2-torus. (The two odd directions generate the cohomology of the 2-torus.)
Elaborating on this point of view with an analysis of T-duality recovers
Witten's prescriptions for Chern Simons theory via topological string theory.
\end{rmk}

\subsection{Closed String States}
Similarly, the closed string state space will be of the form
$$\mathrm{A-model closed states}\otimes \mathrm{B-model closed
states}$$
For the A-model on a symplectic manifold, as usual, we get the cohomology of
that manifold; but again as we are constructing a field tehory on $\R^2$ we
take the de Rahm cohomology complex.
For the B-model, following BCOV's analysis on a CY 3-fold, we consider $$\Ker \partial \subset
\mbf{PV}^{*,*}(\C^2) \simeq
\Omega^{0,*}(\C^2)[\partial_{z_1},\partial_{z_2}]$$
i.e.\ the kernel of a certain operator on the space of poly-vector fields.
\begin{rmk}
  The $\partial_{z_i}$ are odd.
\end{rmk}

\begin{rmk}
  The constructions here are much more homologically involved. Updates coming soon.
\end{rmk}

Observe that the action of $\partial$ on PolyVector fields $$\partial :
\mbf{PV}^{i,*} \rightarrow \mbf{PV}^{i-1,*}$$
and that $$\mbf{PV}^{0,*}\subset \Ker\partial.$$ Now since $\C^2$ is symplectic, we
find$$\mbf{PV}^{1,*}(\C^2) \simeq
\Omega^{1,*}(\C^2)$$ with the restriction of the operator $\partial$ acting as the holomorphic de
Rahm operator. Hence we are considering the complex of
closed\footnotemark\footnotetext{closed in the holomorphic sense} holomorphic
1-forms.

\begin{rmk}
  Following BCOV, we have stipulated that the fields must satisfy a
  differential equation, which inevitably produces a technical nightmare.
  Further, what is worse is that the action functional (coinciding with the
  inverse of $\partial$) is non-local.
\end{rmk}

Also, note that when restricting to poly-2-vector-fields $$\mbf{PV}^{2,*}(\C^2) \supset \Ker\partial$$
$\Ker\partial$ picks out poly-2-vector-fields that are independent of $z$,
hence
$$\mbf{PV}^{2,*}(\C^2) \supset \Ker\partial \simeq \C$$
(the cohomology is $\C$), which is again (almost) trivial. What is more, by the
definition of a propagator in BCOV theory, this does not propagate---hence form
background fields which we'll ignore.

\begin{rmk}
  this statement generalizes the sheafy setting.
\end{rmk}

\subsection{Closed String Field Theory}

Combining this analysis, for our candidate for the closed string field
theory we are led to divergence free vector fields; equivalently, in this case, symplectic
vector fields. Now we opt to replace $\Ker\partial$ by the image
of $\partial$ i.e.\ by $$\Omega^{0,*} - \mathrm{Hamiltonian Vector
Fields}$$

\begin{rmk}
  We have replaced one complex by another that is almost---but not
  quite---quasi-isomorphic: they differ by the cohomology of $\C$, which is
  (almost) trivial.
\end{rmk}

\subsubsection{The Action Functional}

We can now write down the action, for $$\alpha \in
\Omega^*(\R^2)\hat{\otimes}\Omega^{0,*}(\C^2)[1]$$
$$\beta \in \Omega^*(\R^2)\hat{\otimes}\Omega^{0,*}(\C^2)[2]$$
the action is an integral over real 6-dimensional space, cubic in the fields
$$\int_{\R^2 \times \C^2} \beta\bar{\partial}\alpha dz_1 dz_2 +
\int_{\R^2\times \C^2} \beta \partial \alpha \partial \alpha$$

\begin{rmk}
  Observe that the second term of the action can be expressed as
  $$\int_{\R^2\times \C^2} \beta \partial \alpha \partial \alpha =
  \int_{\R^2\times \C^2} \beta\{\alpha,\alpha\}dz_1 dz_2$$
  where the expression $\partial \alpha \partial \alpha$ can be expressed as a
  poisson bracket, hence we see the lie bracket on hamiltonian vector fields
  entering the action functional.
\end{rmk}

\begin{rmk}
  Before, the quantum mechanical system lived on a line (extent of the brane), while this gravitational theory occupies all of space-time.
\end{rmk}

\subsubsection{De-BV-fying the closed string field theory}

\begin{rmk}
  This theory is closely related to a 4-dimensional BF theory.
  TODO more
\end{rmk}

Consider space-time $$\R^2_{x_i}\times\C^2_{z_i}$$ where the subscripts
indicate coordinates. Paying attention to ghost number zero components
$$\beta = \beta_{x_1 x_2} dx_1 dx_2 + \beta_{x_i \bar{z}_j} dx_i d\bar{z}_j +
\beta_{\bar{z}_1\bar{z}_2 d\bar{z}_1d\bar{z}_2}$$
$$\alpha = \alpha_{x_i}dx_i + \alpha_{\bar{z}_j}d\bar{z}_j$$
and considering the action as given above, we see that $\beta$ has $1$-form gauge
symmetry, while $\alpha$ has $0$-form gauge symmetry.
\begin{rmk}
  As in the previously alluded to $4$-dimensional BF theory, $\beta$---having
  $1$-form gauge symmetry---also has secondary gauge symmetry, which is why we
  find a large complex. TODO more.
\end{rmk}

Analysing $\alpha$, as a Hamiltonian vector field on $\C^2$, we see the
term $\alpha_{\bar{z}_j}d\bar{z}_j$ is the Beltrami differential, where we
deform $\C^2$ as a symplectic surface. The term $\alpha_{x_i}dx_i$ is a
connection on $\R^2$ valued in Hamiltonian vector fields; the terms together
imply that we have a flat bundle of $\C^2$'s over $\R^2$. In fact, the
equations of motion determine an integrable deformation of complex structure.
Analysing $\beta$ in the action functional, it only appears linearly. This
implies they're just lagrangian multiplier fields in enforcing the equations
resulting in the integrable deformation of complex structure + flatness of the
bundle of $\C^2$'s over $\R^2$.
\begin{rmk}
  This final statement is analogous to BF theory where $B$ is a lagrangian
  multiplier and $F$ is zero, hence why we see in the next section that the
  phase space is a cotangent bundle.
\end{rmk}

\subsubsection{Phase Space}
If we put this on $$\R \times S^1 \times X$$ where $X$ is homolomorphic
symplectic, then the phase space is $$T^*(\mathrm{Moduli of
holomorphic-symplectic surfaces fibered over} S^1 + \mathrm{flat
connection})$$
\begin{rmk}
  If we work in perturbation theory near $X$ we will find the above moduli
  space near this configuration.
\end{rmk}

\subsection{Closed String Fields}

Given a closed string field, we should obtain a deformation of the gauge theory
on the brane. For example, if the brane is on the line $x_1$ and the closed
string  field is $$dx_1 z_1^k z_2^l$$ we ought to obtain a deformation of our
quantum mechanical system on the line $x_1$. We can identify this deformation
by deforming the action to $$\int Tr \phi_1 d_A \phi_2 + \int dx_1 Tr \phi_1^k
\phi_2^l$$ by adding on a Hamiltonian. Many matrix models of hamiltonians can
be engineered by these backgrounds.

\begin{rmk}
  Geometrically, this just amounts to saying that the $\C^2$ has a connection,
  allowing parallel transport.
\end{rmk}

\subsection{Operators in the gauge theory, and the large $N$
limit}

TODO more
In any theory where we find the de Rahm complex as a space of fields, all the
local operators cannot have any derivatives (in $X$); anything involving a
derivative in $X$ can be killed by the BRST operator.\par

It follows that we will only care about values of fields at a point, but also
of anti-fields etc.\

\subsubsection{Ghost $0$ operators}
These will be generated by elements of the form $$Tr(p(\phi_1,\phi_2))$$ where $P$ is a
non-commutative polynomial\footnotemark.\footnotetext{Here we are appealing to invariant theory, where
the algebra of invariant matrices is generated by elements of the above form.}
As $N \lim \infty$, there are no relations, these are freely generated.
\subsubsection{Ghost $-1$ operators}
Ghost number $-1$ operators will be generated by elements of the form
$$Tr(\psi(p(\phi_1,\phi_2))$$
  where $$\psi = A^*$$
  is the anti-field to the field given by a connection $A$.
  TODO write up rest of fields

\subsubsection{Algebra of local operators on the brane}
By cyclic symmetry of the trace $$\int Tr(\phi_1[A,\phi_2]) = \int
Tr(A[\phi_1,\phi_2])$$
Since the BRST operator acts on the anti-field of the connection, and it
appears at $\phi_1,\phi_2$, it turns it into a commutator, hence
$$Q(Tr(\psi p(\phi_1,\phi_2)) = Tr([\phi_1,\phi_2]p(\phi_1,\phi_2))$$

\begin{cor}
  In $Q$-cohomology, expressions involving words in $\phi_1,\phi_2$ can be
  commuted past one-another. We have a basis of ghost \#$0$ operators given by
  $$Tr(\phi_1^k\phi_2^l)$$
  and ghost \#$-1$ operators are given by\footnotemark $$Tr(\mathrm{Sym}\{\psi \phi_1^k
    \phi_2^l)$$
    Similarly, the classical local operators are $$\mathbf{S}^*(\C[z_1,z_2]
    \oplus \C[z_1,z_2][1])$$
    where $\mbf{S}^*$ denotes the dual of the symmetric algebra.
\end{cor}
\footnotetext{$\mathrm{Sym}\{\cdots\}$ denotes the symmetrized expression:
i.e.\ the sum of permutations of factors of the product of the expression}

\begin{rmk}
  The ghost \#$0$ operators in the corollary above correspond on the string-gravitational side to polynomial observables
  $$z_1^kz_2^l$$
\end{rmk}

\subsection{Operators in the gravitational theory}
TODO write up full argument.
Similarly, we can build local operators through a limited series of operations:
we can evaluate $\theta$ and $\alpha$ at a point, we can differentiate etc.\ By
a standard cohomological argument, only the zero-form parts of $\alpha$ and
$\beta$ will yield physical operators in the $Q$-cohomology, while only the
$z_1$ and $z_2$ derivatives can contribute, as the other derivatives can be
cancelled by the image of $Q$.\par

The ring of local operators is generated by the operators of ghost \#$1$: $$\alpha \mapsto
\partial_{z_1}^k \partial_{z_2}^l \alpha$$
and operators of ghost \#$2$:
$$\beta \mapsto \partial_{z_1}^k \partial_{z_2}^l \beta$$
These operators have the opposite transformation properties as those from the
gauge theory.\par

We find similarly that the ring of classical local operators is given by
$$C^*(\C[z_1,z_2] \ltimes \C[z_1,z_2][1])$$
Chevalley cochains of a semi-direct product of (shifted) polynomial rings. The
first $\C[z_1,z_2]$ in the semi-direct product, via the equations of motion,
corresponds to the $\alpha$ field and is a lie algebra with poisson bracket,
while the second $\C[z_1,z_2][1]$ as the adjoint module acted upon by the
first, corresponds to the $\beta$ field.

\begin{rmk}
  Recall again that think of $\alpha$'s as
  hamiltonian vector fields, while the $\beta$'s are functions admitting an action
  of hamiltonian vector fields, when  we can consider $\alpha \in \C[z_1,z_2]$ and $\beta \in
  \C[z_1,z_2][1]$.
  The action is then given by
  $$[\alpha, \beta] = \{\alpha,\beta\}$$ where the poisson bracket is given by
  the constant poisson tensor $dz_1dz_2$. In particular, $[\alpha,\alpha]$ is
  just the poisson bracket while $[\beta,\beta]$ commutes.
\end{rmk}

\subsection{Koszul Duality}
  At this stage we can see that the koszul dual of these two algebras of local
  operator is nearly koszul dual, i.e. the koszul dual of the above gravity
  operators is given by $$\mbf{U}(\Hom(\C^2)\oplus\Hom(\C^2)[1]).$$
  Comparing this to the gauge theory operators, which are
  $$\mbf{S}^*(\C[z_1,z_2]\oplus\C[z_1,z_2][1])$$ we find that the latter is
  really the classical i.e.\ commutative limit of the former.
\subsubsection{Classical free gravity and the commutative limit}
On the gravitational side, we can interpret the passage to the commutative
limit as discarding any interactions; these interactions produced the lie
bracket. Hence via the passage to \textit{classical free
gravity} we see the effect of koszul duality $$\textrm{Classical free gravity} \leftrightarrow^{\textrm{KD}} \textrm{Classical gauge theory}$$
It follows that moving to first order away from the classical limit in the
gauge theory will be necessary to match up both sides via Koszul
duality\footnotemark.

TODO define free (quadratic) gravity.
\footnotetext{
  Going further, in the planar limit, $\hbar$ on the RHS will match up with the interaction
  strength on the LHS. For example, a two-loop diagram from the gauge theory
  will correspond to a genus-two string interaction. Correspondingly, passage
  to quantum gravity, i.e.\ non-trivial $\hbar$
  on the gravitational side will take us beyond the planar limit and manifest
  via traces of operators on the gauge theory side. We'll need to include a
  parameter in the OPE of local operators: e.g.\ the OPE of two single-trace
  operators will yield a spectrum of operators of various traces---here we will
  see the appearance of this parameter.
}

\subsubsection{First order corrections}

Under the koszul duality correspondence we have
$$z_1^k z_2^l \leftrightarrow Tr \phi_1^k\phi_2^l$$
and we want the poisson bracket to match up to a non-trivial first-order
quantum correction $$\{z_1^kz_2^l,z_1^mz_2^n\} \leftrightarrow [Tr
\phi_1^k\phi_2^l, Tr \phi_1^m\phi_2^n]$$

This we can compute by a diagram linking by an edge the two sub-diagrams corresponding to
each trace, which produces a match with the poisson bracket.
\image{firstorder.png}

\subsubsection{Second order corrections: $\hbar^2$}
Attempting to work to order $\hbar^2$, we can appeal to symmetry: the algebra
of trace of operators is an even quantization; namely, sending $$\hbar \mapsto
-\hbar$$ and simultaneously reversing the product yields an equivalence. Thus
to order $\hbar^2$, we find no corrections.

\subsubsection{Third order corrections}
Restricting for the moment to ghost \#$0$ operators, we need to account for third order diagrams:
\image{thirdordera.png}
\image{thirdorderb.png}
which are ribbon graphs in the large-$N$ limit.

The first diagram above gives a cubic expression in traces, while the second
gives a quadratic expression (the degree coincides with the number of
boundaries of the diagram).\par
\begin{rmk}
  We see from the first diagram above that koszul duality switches degrees: a
  expression taking three generators (the cubic trace) gets mapped to one with
  two generators. In the $A_{\infty}$ language, the first diagram yields an
  $m3$ $A_{\infty}$ operation, while the second yields an $m_2$ operation.\par

  Note also that the cubic expression in traces is symmetric,
  as we expect: a poisson bracket on an exterior algebra ought to be symmetric.
\end{rmk}

In order to match these with diagrams on the string-gravity side, we need to
understand the $A_{\infty}$ structure on the algebra of cochains
$$C^*(\C[z_1,z_2]\oplus\C[z_1,z_2][1]).$$

Given operators $$\mathcal{O}_{k,l} = \partial_z^k\partial_k^l\alpha$$ to
compute the $A_{\infty}$ structure we'll need to match these two diagrams: the
first maps to a sum over all feynman diagrams

TODO draw feynman diagram
%\image{m3feynmandiagram}

yielding a $3 \rightarrow 2$ $A_{\infty}$ operation, which on the koszul dual
side maps to a $2 \rightarrow 3$ operation, while the second diagram maps to a
sum over all feynman diagrams.

TODO dram feynman diagram
%\image{m2feynmandiagram}

\begin{rmk}
  In practice in trying to work more generally, in order to show uniqueness,
  we'd need to identify the underlying algebra and show (homologically) that it
  has only one deformation.
\end{rmk}

\begin{rmk}
  throughout, we could have treated $lie{gl}_N$ in two ways: as a lie algebra,
  or as a reductive group. Taking invariants with respect tothe latter choice
  is an exact functor, hence obviating the need for Chevalley cochains and
  discarding all ghosts.\par

  Including all of cyclic cohomology would match up with
  higher degree polynomial invariants in the ghosts: i.e.\ including more
  topological data in our gravitational theory.
\end{rmk}

\begin{rmk}
  On the gravity side we did not choose a brane: we just worked with string
  states. On the gauge theory side we picked a brane of the form a point cross
  a lagrangian. By choosing a different $B$-brane, we move away from the
  structure of an associative algebra to a more general factorization algebra.
  Koszul duality should then extend to this more general setting. For example,
  if we had two topological directions (instead of one), then we would expect
  to obtain an $E_2$ algebra and hence $E_2$ koszul duality
\end{rmk}

\begin{rmk}
  We could also study $D_0$ branes $\R^2\times \C^4$ from this point of view.
\end{rmk}

\end{document}
